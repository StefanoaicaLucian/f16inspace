\section{A Brief History of Computer Graphics}
First of all, what is \textit{computer graphics}? The term first appeared in 1960 and it was made-up by William Fetter who was, at the time, a computer graphics researcher for Boeing. The term refers to a subfield of computer science which deals with image data representation and manipulation using computers.\\

The first computer able to do graphics is the Whirlwind computer. Its development was started in 1945 at MIT by a team of computer scientists led by Jay Forrester. The purpose of this computer was to make aircraft tracking possible on a large oscilloscope screen via a graphical application. Although the aircraft tracking application was the first application in the field of computer graphics it was not interactive. It could only display the real time positions of the tracked aircrafts. The first interactive graphical application was Tennis For Two and it was created by William Higinbotham because he wanted to kill the boredom of the visitors of the Brookhaven National Laboratory.\\

In 1959 the TX-2 computer emerged. This computer was used by Ivan Sutherland to program the Sketchpad, a tool for creating very precise engineering drawings. The software offered its users the possibility to draw lines and circle arcs. The lines could then be made perfectly parallel or perpendicular in order to fit the users drawing needs. Sketchpad is known to be the first graphical user interface or GUI in short form.\\

In 1966 Ivan Sutherland made yet another contribution to the field of computer graphics by inventing the Sword of Damocles the first computer controlled head mounted display. This device displayed two stereoscopic images of the same wire-frame mesh. Two decades later NASA would use his methods in virtual reality research.\\

Very soon after, in 1970 actually, the field of computer graphics was upgraded by Henri Gouraud, Jim Blinn and Bui Tuong Phong. The first added the Gouraud shading model and the other two added the Blinn-Phong shading model. In 1978 Jim Blinn also added bump mapping to the computer graphics field.
\section{A Short History of Particle Systems}
The term \textit{particle system} was coined in 1982 by William T. Reeves who was a computer graphics researcher for Lucasfilm Ltd., a film production company known for films like the Star Trek and Star Wars franchises. But what is a particle system? Here's the original definition:\\

"A particle system is a collection of many minute particles that together represent a fuzzy object. Over a period of time, particles are generated into a system, move and change from within the system, and die from the system."\\

--William T. Reeves--acm Transactions On Graphics--April 1983--Vol. 2, No. 2\\

At Lucasfilm Ltd. Reeves helped to develop the wall of fire effect which occurred every time the Genesis Device was used in the film Star Trek II: The Wrath of Khan. This fictional device had the amazing ability to transform any lifeless planet into a perfectly habitable place. Though this is not the first time when particle systems have been used in the computer industry, this particular effect helped to coin the term \textit{particle systems}.\\ 

The first particle systems were implemented in the earliest video games where this concept was used to portray exploding spaceships which left behind many little dots or lines. For example "Spacewar!" which appeared in 1962 and "Asteroid" which appeared later in 1978 had such effects implemented. Since their dawn till the very present particle systems have been used to implement various interesting phenomena such as fire, smoke or waterfalls that would be otherwise be very difficult to implement if not impossible.

\section{The Role of Particle Systems in Computer Graphics}