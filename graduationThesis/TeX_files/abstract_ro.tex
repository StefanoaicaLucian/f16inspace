\thispagestyle{plain}

\begin{center}
	\textbf{Abstract}
\end{center}

Grafica pe calculator ne \^ \i mbun\u{a}t\u{a}\c{t}e\c{s}te via\c{t}a de zi cu zi oferindu-ne unelte pentru proiectare asistată de calculator si mijloace de divertisment. Sistemele de particule \^ \i mbog\u{a}\c{t}esc grafica pe calculator f\u{a}c\^{a}nd posibil\u{a} redarea de fenomene naturale interesante cum ar fi focul sau fumul. Interfa\c{t}a de programare de aplica\c{t}ii prezentat\u{a} in aceast\u{a} lucrare de licen\c{t}\u{a} \^{i}mpreun\u{a} cu concepte legate de implementare este defapt o mica libr\u{a}rie ce con\c{t}ine c\^{a}teva efecte grafice bazate pe sisteme de particule. Lucrarea este structurat\u{a} in patru capitole si con\c{t}inutul fiecarui capitol este descris pe scurt mai jos.\\

Capitolul 1 ofer\u{a} un scurt istoric al graficii pe calculator \c{s}i al sistemelor de particule. Acest capitol specifica de altfel \c{s}i rolul pe care \^{i}l joac\u{a} sistemele de particule in grafica pe calculator \c{s}i de ce sunt ele necesare.\\

Capitolul 2 descrie ni\c{s}te detalii tehnice legate de sisteme de particule. Dup\u{a} ce cite\c{s}te acest capitol un programator ar trebui sa aib\u{a} deja o idee destul de bun\u{a} despre cum ar trebui implementat un astfel de sistem intr-o aplica\c{t}ie grafic\u{a}.\\

Capitolul 3 prezinta toate efectele grafice bazate pe sisteme de particule din librarie si ofera detalii legate de implementarea lor.
De asemenea mai ofera si detalii legate de structura librariei si a unei aplicati ce foloseste libraria. Mai ofera si niste capturi de ecran ce demonstreaza efectele din librarie.\\

Capitolul 4 discuta niste directii de lucru viitoare care ar trebui sa imbunatateasca performanta de redare a librariei. Mai ofera si o scurta descriere a experientei mele de invatare.