\section{How To Design Particle Systems}
From an object-oriented programming point of view a \textbf{particle system} is a simple container object which manages all the particles inside it. This container has an interface which programmers can use to control the system and the particles inside it. But how would one design and implement such a container?\\

Here are a few steps which can be carried out in order to obtain an object-oriented model of a problem. In order to highlight some of the base components of an object-oriented system a programmer should write down a detailed description of the problem which the system is supposed to solve (here the problem is how to "tell" a computer what a particle system is and how does it behave). After that, the programmer should iterate through the description and identify all the nouns, adjectives and verbs. The nouns represent potential class names, the verbs represent potential class attributes and the verbs represent potential class methods. The next step is to associate class names with class attributes and class methods.\\

In the case of the particle systems problem, a particle could be a class, and also, the particle system itself could be a class. But these classes should only be generic classes, therefore they should only contain information which is common to all the particles and all the particle systems respectively. This classes will never be instantiated, instead they will serve as parent classes to other concrete classes which will hold specialized information about specific particles and particle systems. Some particles may move differently than others and some particle systems may emit more than one type of particle. But how do particle systems actually work? Every time a particle system is being rendered a few things happen. Being a simple container which manages its contained objects a particle system starts off by being completely empty. At each rendering cycle new particles are created and added to the particle system. The list of contained particles is checked for "dead" particles (these are particles that have exceeded their expected duration) in order to remove them. Each particle in the system is first placed in the origin of the system and has its next position calculated with regards to its old position and other attributes like direction, speed or acceleration.