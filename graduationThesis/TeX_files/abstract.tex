\thispagestyle{plain}

\begin{center}
	\textbf{Abstract}
\end{center}

The application programming interface which is presented in this thesis, alongside implementation related concepts, is actually a small library packed with a couple of graphical effects based on particle systems. The thesis is composed out of four chapters and the content of each chapter is shortly described bellow.\\

In the \textit{Introduction} chapter, a brief history of computer graphics and particle systems is given. This chapter also specifies the role that particle systems play in computer graphics and why are they necessary.\\

The \textit{Particle Systems} chapter obviously describes some technical details about particle systems. By the end of this chapter a programmer should already have an idea about how to implement such a system in a graphical application.\\

The \textit{Application} chapter presents all the particle based graphical effects in the API and gives implementation details about them. It also gives details about the structure of the API and that of the application which uses the API. Besides this it demonstrates the API's use with a couple of screen-shots.\\

Very few software pieces are perfect at their first implementation. The \textit{Conclusion} chapter emphasizes some changes which can be made in order to improve the performance of the API and gives a short description of my learning experience.