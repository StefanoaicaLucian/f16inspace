\thispagestyle{plain}

\begin{center}
	\textbf{Abstract}
\end{center}

Computer graphics improves our daily lives by offering computer aided design tools and means of entertainment. Particle systems are what enrich computer graphics by enabling the possibility to render fire, smoke and other interesting natural phenomena using computers. The application programming interface presented in this thesis, alongside implementation related concepts, is actually a small library packed with a couple of graphical effects based on particle systems. The thesis is composed out of four chapters and the content of each chapter is shortly described bellow.\\

Chapter 1 gives a brief history of computer graphics and particle systems. This chapter also specifies the role that particle systems play in computer graphics and why are they necessary.\\

Chapter 2 describes some technical details about particle systems. By the end of this chapter a programmer should already have an idea about how to implement such a system in a graphical application.\\

Chapter 3 presents all the particle based graphical effects in the API and gives implementation details about them. It also gives details about the structure of the API and that of the application which uses the API. Besides this it demonstrates the API's use with a couple of screen-shots.\\

Chapter 4 discusses some future work directions which should enhance the rendering performance of the API. It also gives a short description of my learning experience.